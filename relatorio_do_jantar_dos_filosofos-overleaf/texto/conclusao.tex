A abordagem utilizando mutex e semáforos se mostrou eficiente para resolver o problema do jantar dos filósofos, evitando deadlocks. No entanto, melhorias podem ser feitas para reduzir a possibilidade de starvation, como integrar semáforos para limitar acessos simultâneos. Reflexão sobre os aprendizados obtidos inclui a importância da sincronização em sistemas concorrentes e considerações sobre as abordagens mais eficazes. Possíveis trabalhos futuros envolvem uso de aprendizado de máquina para otimizar prioridades, simulações mais complexas com N variável ou integrações com frameworks distribuídos como MPI.