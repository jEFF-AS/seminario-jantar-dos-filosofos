O trabalho realizado tem como objetivo encontrar uma possível solução para o problema denominado jantar dos filósofos. Este tem a finalidade de estudar e implementar o método mutex para resolver esse problema, discutindo as vantagens, limitações e complexidade desta abordagem. O problema do jantar dos filósofos trata da coordenação de processos competindo por recursos compartilhados, sendo uma metáfora clássica para ilustrar questões de sincronização em sistemas operacionais multitarefa. Neste seminário, focamos na abordagem com mutex, que garante exclusão mútua e evita deadlocks por meio de estratégias de aquisição de recursos. A implementação em Python demonstra como threads podem simular os filósofos, alternando entre estados de pensamento e alimentação, enquanto locks representam os talheres compartilhados. Essa abordagem é escolhida por sua simplicidade e eficiência em ambientes concorrentes, embora exija cuidados para mitigar riscos como starvation. O estudo contribui para a compreensão de mecanismos de sincronização, com aplicações em programação paralela e sistemas distribuídos. Além disso, são exploradas abordagens complementares, como estatística, teoria dos grafos e SAT/CSP, para fornecer uma visão integrada dos métodos disponíveis na literatura, permitindo uma comparação aprofundada das soluções em termos de viabilidade prática e teórica.