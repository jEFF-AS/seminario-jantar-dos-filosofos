A sincronização de processos é essencial em sistemas operacionais multitarefa. Mutex e semáforos são mecanismos utilizados para garantir exclusão mútua e coordenação entre threads. O mutex permite que apenas um processo acesse o recurso por vez, enquanto os semáforos permitem um controle mais flexível sobre múltiplos acessos. Em contextos de programação concorrente, esses primitivos evitam condições de corrida, onde múltiplas threads modificam dados compartilhados simultaneamente, levando a inconsistências. Mutex, em particular, é um semáforo binário que opera com operações de lock (adquirir) e unlock (liberar), garantindo que apenas uma thread detenha o recurso crítico. Semáforos gerais, por outro lado, podem contar acessos permitidos, sendo úteis em cenários com múltiplos recursos idênticos. No problema do jantar dos filósofos, esses mecanismos são aplicados para gerenciar o acesso aos talheres, prevenindo deadlocks (bloqueio mútuo onde threads esperam indefinidamente) e starvation (onde uma thread nunca acessa o recurso).

\subsection{Descrição do Problema}
Apresentação do problema conforme proposto por Dijkstra. Cinco filósofos estão sentados em uma mesa redonda para jantar. Cada filósofo tem um prato com espaguete à sua frente. Cada prato possui um garfo para pegar o espaguete. Para comer, cada filósofo precisa de dois garfos. A dinâmica é: eles podem alternar entre pensar e comer, mas só podem comer se tiverem dois garfos. Os garfos são compartilhados entre filósofos adjacentes, formando uma configuração circular. Sem sincronização adequada, pode ocorrer deadlock se todos os filósofos pegarem o garfo esquerdo simultaneamente, criando um ciclo de espera. Além disso, starvation pode afetar filósofos se vizinhos monopolizarem os garfos. O problema ilustra desafios reais em sistemas operacionais, como alocação de recursos em multiprocessadores ou gerenciamento de I/O em redes distribuídas. A solução deve garantir progresso (pelo menos um filósofo avança), ausência de deadlock e fairness (todos eventualmente comem).