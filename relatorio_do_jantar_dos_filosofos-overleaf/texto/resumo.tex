O problema do jantar dos filósofos é um clássico da computação concorrente, proposto por Dijkstra, que busca representar os desafios de sincronização e alocação de recursos em sistemas paralelos. Neste trabalho, são analisadas diferentes abordagens para sua resolução, incluindo técnicas estatísticas, uso de semáforos/mutex, teoria dos grafos e resolução via problemas de satisfação de restrições (SAT/CSP). Com ênfase na abordagem com semáforos/mutex, é apresentada a implementação em Python utilizando mutex para evitar deadlocks e condições de corrida, complementada por uma variação com semáforos para maior robustez. São discutidas as implementações, resultados e análises comparativas, destacando vantagens como simplicidade e robustez, limitações como possível starvation e complexidade moderada. A solução adotada utiliza locks para representar os talheres, com estratégias de aquisição ordenada ou controlada para prevenir bloqueios mútuos, garantindo uma análise abrangente das implicações teóricas e práticas.