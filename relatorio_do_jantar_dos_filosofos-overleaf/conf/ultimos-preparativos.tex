\renewcommand\Authsep{, \quad }
\renewcommand\Authand{, }
\renewcommand\Authands{, \quad }

\renewcommand*{\Affilfont}{\normalsize\normalfont}
\renewcommand*{\Authfont}{\scshape}

\title{%
    \vspace{-15mm}%
    \fontsize{20pt}{26pt}%
    \selectfont\textbf{\tituloportugues} \\
    \vspace{1ex}
    \fontsize{14pt}{20pt}\selectfont\textbf{\color{gray}\subtitulo}%
} % Título.
\date{}

% Comando para adicionar fonte de figuras e semelhantes:

\newcommand{\source}[1]{\captionsetup{singlelinecheck=false,justification=justified}\caption*{\footnotesize \noindent Fonte: {#1}}}

% Para resolver conflito do verbatim:

\usepackage{tikz,everypage}
\makeatletter
\AtBeginDocument{%
  \AddEverypageHook{%
    \small    
    \catcodetable  \catcodetable@latex %
% Descomentar abaixo para colocar números nas linhas:
%    \begin{tikzpicture}[remember picture,overlay]
%      \path (current page.north west) --  (current page.south west)
%            \foreach \i in {1,...,\fakelinenos}
%               { node [pos={(\i-.5)/\fakelinenos},
%                  xshift=\fakelinenoshift, line number style] {\i} }  ;
%    \end{tikzpicture}%
  }
}
\makeatother    
\tikzset{%
  line numbers/.store in=\fakelinenos,
  line numbers=72,
  line number shift/.store in=\fakelinenoshift,
  line number shift=5mm,
  line number style/.style={text=gray},
}